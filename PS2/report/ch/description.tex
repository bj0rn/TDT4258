\section{Description And Methodology}


\subsection{Initial Setup}
To begin with, peripherals such as buttons, LEDs, DAC, and timers needed to be configured. As C programs can control hardware directly, one only needs to use pointers to refer to memory-mapped I/O registers. A lot of these were provided in the efm32gg.h-file where many of the registers were defined with macros by using volatile uint32_t pointer variables, so in the initial setup it was simply a matter of dereferencing these and writing the appropriate values.


\subsubsection{Interrupt handlers}
The procedure for setting up interrupt handlers will vary depending on the hardware used. In this case, as the exception vectors were already filled out in the startup code, all that needed to be done to enable interrupt generation and interrupt handling was to write the appropriate value to the ISER0-register, whic was done in the setupNVIC()-function. The code for enabling interrupts for the GPIO-pins (odd and even) and TIMER1 was as follows. 

\begin{lstlisting}

void setupNVIC()
{
 	*ISER0 = 0x1802; /*Enable interrupt handling for odd and even GPIO pins and TIMER1*/
}

\end{lstlisting} 
This corresponds to writing bit 1, 11, and 12 in the register, which are the appropriate bits for enabling interrupts for the mentioned exception handlers.

\subsubsection{Hardware timers}
There are several hardware timers that can be used, depending on the purpose (calculation-intensive operations, low energy demnads, etc...). In the initial setup, 


\subsection{Sound Wave Synthesis}
Sound is realised through sound waves which is created by oscillations. "An oscillator generates a consistent, repeating signals". These consistent signals can be used to create waves at various frequencies. Sound is actually the properties of the waves generated, with respect to frequency, amplitude, and period. The frequency 

The waves are generated  


In case of this assignment, we have decided to take two approaches for generating sound. The first one is implementing an synthesiser in software. This synthesiser is created by using square waves. The square wave is generating the approximations of the frequency, for various tones. The tones are as follows, A, B, C, D, E, and F. The waves are realised by writing to the DAC with the various frequencies, thereby approximating the different tones. The period of the tone is determined by a counter.

\subsection{Sound Sampling}
Sound can also be generated through pre computed samples. With this approach the sound waves are generated on some other platform. The samples produced are approximations to already existing sound waves.


\subsection{Energy Optimization}
In order to decrease the energy consumption of our program we have taken various different methods to reduce the consumption. Instead of using the proposed timer\cite{compendium}, we are instead using the low energy timer, LETIMER0 \cite{EFM32GG-rm}. This timer runs at a lower frequency, but require an low frequency oscillator in order to be of any value. The advantage with this timer is that it uses less energy and is able to run in EM2.  



\subsection{User Control}

In order to let a user be able to interact with our setup, the buttons on the gamepad were configured so that different sounds would play when one of them were pressed. Each button was configured to play one specified sound or melody. This was done by reading the value of the $GPIO\_PC\_DIN$ register, and writing this value to a function $play\_melodies(int)$ that would take the $GPIO\_PC\_DIN$ value as an input parameter and play the corresponding melody. One of the challenges in this was to get the DAC to output the entire melody, and not just for the duration in which a button was pressed. To solve this, a pointer variable was used to point to the correct sample array and let the timer run for the entire length of the array in order to play the whole melody. An alternative approach was to declare a global variable and set it to the value of the $GPIO\_PC\_DIN$ register in each of the GPIO-handlers, and use a switch-statement in the low-energy timer function to play the corresponding melody, but the former approach was prefered due to its energy efficiency advantages.

Our setup was as follows:

\begin{table}[ht]
\caption{Buttons and corresponding melodies}
% title of Table
\centering
% used for centering table
\begin{tabular}{c c}
% centered columns (4 columns)
\hline
\hline %inserts double horizontal lines
Button & Melody \\ [0.5ex]
% inserts table
%heading
\hline
SW1 & Mario \\
SW2 & Shoot \\
SW3 & Hit dealt \\
SW4 & Hit received \\
SW5 & Battlefield intro \\
SW6-SW8 & Simple *beep*-noises \\
% [1ex] adds vertical space
\hline
%inserts single line%
\end{tabular}
\label{table:nonlin}
% is used to refer this table in the text
\end{table}











