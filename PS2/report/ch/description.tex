\section{Description And Methodology}
This section describes the various techniques employed to develop a sound synthesiser in software. The development of the program have taken an iterative approach where the first stages focused on getting everything up and running. Then the program was further developed to support more and more features. The last stage focused on energy optimizations. {\bf rewrite}


\subsection{Initial Setup}
In this assignment the following peripherals need to be configured, buttons, LEDs, DAC, and some timer. The buttons are needed to allow interaction with the board. The buttons are configured to play different sounds when pushed. 




\subsection{Sound Wave Synthesis}
Sound is realised through sound waves which is created by oscillations. "An oscillator generates a consistent, repeating signals". These consistent signals can be used to create waves at various frequencies. Sound is actually the properties of the waves generated, with respect to frequency, amplitude, and period. The frequency determines the tone of the sound, the amplitude the strength, and the period the duration of the sound.  

There are various different approaches for generating these sound waves. The different waves have slightly different properties in regard to sound. To name a few different waves we have the sine wave, sawtooth wave, triangle wave and the square wave. These waves produces different sound characteristics {\bf more on this ? Sources and shit like that}. 

In case of this assignment, we have crated two different kind of waves, the square wave and the sawtooth wave. The square wave is created by calculating discrete samples based on the tone frequency and the oscillator frequency. Oscillator frequency divided by the tone frequency decides how often the values to the DAC should alternate, thereby producing an approximation of the note. The different notes supported by the program is as follows, A, B, C, D, E, F, and H.


\begin{figure}[H]
  \centering
  % Trim er [left bottom right top]
  \includegraphics[clip, trim=0cm 0cm 0cm 0cm, width=4cm]{fig/square_wave.pdf}
  \caption{Square wave}
\end{figure}

\begin{figure}[H]
  \centering
  % Trim er [left bottom right top]
  \includegraphics[clip, trim=0cm 0cm 0cm 0cm, width=4cm]{fig/sawtooth.pdf}
  \caption{Sawtooth wave}
\end{figure}



\subsection{Sound Sampling}
Sound can also be generated through pre computed samples. With this approach the sound waves are generated on some other platform. The samples produced are approximations to already existing sound samples. The samples are modified slightly to fit the microcontroller. Due to size limitations on the board, 8 bit samples with 8000Hz were generated. This allowed us to be able to play song with some length. In the first approach we generated samples with a sample frequency of 48000Hz. This sample rate produced sounds with relatively high sounds quality, but required a lot of storage for the songs. This limited the lenght of the songs we were able to play.  



\subsection{Energy Optimization}
The EFM32GG microcontroller has a rich set of features regarding improving energy efficiency, where the main focus is turing of components that are not in use, or tuning their performance. In case of this assignment, we have utilised many of these techniques to improve the power consumption of the program. Where our main goal has been trying to exploit the various energy levels that the microcontroller had to offer. Preferable being able to achieve EM2, and turn of every component currently not being used. 

Various techniques were employed to reach this goal. As it turned out the achieving deep sleep and sleep on exit proved to be more complicated than first imagined. During deep sleep mode the high frequency oscillator was turned off. This oscillator was used to clock both the DAC and the timer. When entering EFM32 this produced some unexpected bugs. On entering EM2 the program started to behave non-deterministic. As it turned out, both the DAC and the timer was clocked by the high frequency oscillator. This directly affected the interrupts produced by the timer.

This problem was solved by introducing the low energy timer. This timer is able to run as low as to EM2 and EM3 depending on the oscillator used. In case of producing sound, the 32,768 KHz oscillator were used. This timer replaced the old timer, and only caused some minor changes to made in the already existing code. In particular, the frequency used to generated was changed to accommodate the new oscillator frequency. Instead of  using a sample rate of 48000, the new 
new oscillator were able to produce 32768 interrupts every second. This implied a maximum sample rate of 32768.

%\begin{lstlisting}
%void setupLowEnergyTimer(){
%    *CMU_OSCENCMD = (1 << 6); /* Enable the low frequency oscillator*/
%    *CMU_HFCORECLKEN0 |= (1 << 4); /* Enable LE clock*/
%    *LETIMER0_CTRL |= (1 << 9); /* Set COMP0 as top register*/
%    *CMU_LFACLKEN0 |= (1 << 2); /* Enable low energy timer0*/
%    *LETIMER0_TOP = 1;  /* Set TOP to 1 */
%    *LETIMER0_IEN = 1;  /* Enable interrupts */ 
%    *LETIMER0_CMD = 1;  /* Start the timer */
%}

%\end{lstlisting}

This produced a lower energy consumption, however DAC still caused problems. Since the DAC were configured to continuously pull the DAC registers. The were some problems when entering the low energy mode. The entering of energy mode caused the DAC to continue producing static sound. As it turns out, configuring the DAC with continuous mode will not be able to maintain the voltage levels when entering deep sleep. This cause the voltage levels to fluctuate, and create a static background sound. Fortunately, the DAC also support another mode, sample/hold mode. During sample/hold mode, the DAC core converts on a triggered conversion and then holds the output in a sample/hold element. When not converting, the DAC is turned of between sample, which reduces the power consumption. The sample/hold element will hold the element for a certain time without a refresh conversion\cite{EFM32GG-rm}. Fortunately, this holding time is sufficient in order to fix our bug, and on the same time further decrease the energy consumption. This change from continuous mode to sample/hold mode is achieved by changing the value written to the $DAC0\_CTRL$ register to 0x50014. 


%\begin{lstlisting}
%void setupDAC(){
%    /*...*/
%    *DAC_CTRL0 = 0x50010;   /*The old value*/
%    *DAC_CTRL0 = 0x50014;   /*The new value enabling hold/sample mode*/
%    /*...*/
%}
%\end{lstlisting}

These techniques allow the code to enter deep sleep mode, and also allow utilisation of sleep on exit. This is was accomplished by writing the value 6 to the SCR register setting the deep sleep and deep sleep on exit bit. Then entering EM2 is achieved by calling the WFI instruction.

These techniques combined proved to be very beneficial regarding the energy efficiency. However, we could do even better. During the execution of our program. Most of the functionality were idle. The high frequency oscillator was turned of during deep sleep, however the DAC and the energy timer would still consume power. To allow the program achieve even better decrease in power consumption. We implemented functions that could enable/disable these components. These components we only enabled when needed and disable during idle mode. 



 

  





   



  
       

 








\subsection{User Control}

\subsection{User Control}

In order to let a user be able to interact with our setup, the buttons on the gamepad were configured so that different sounds would play when one of them were pressed. Each button was configured to play one specified sound or melody. This was done by reading the value of the GPIO_PC_DIN register, and writing this value to a function play_melodies(int) that would take the GPIO_PC_DIN value as an input parameter and play the corresponding melody. One of the challenges in this was to get the DAC to output the entire melody, and not just for the duration in which a button was pressed. To solve this, a pointer variable was used to point to the correct sample array and let the timer run for the entire length of the array in order to play the whole melody.

Our setup was as follows:

SW1: Mario
SW2: Shoot
SW3: Hit dealt
SW4: Hit received
SW5: Battlefield intro
SW6 - SW8: Simple *beep*-noises at different frequencies (specified in appendix)













