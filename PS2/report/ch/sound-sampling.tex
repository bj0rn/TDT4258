
Sound can also be generated through pre computed samples. With this approach, the sound waves are generated on some other platform. The samples produced are approximations of already existing sound samples. The samples are modified slightly to fit the microcontroller. Due to size limitations on the board, 8 bit samples at 8000Hz were generated. This decreased the sound quality, but allowed us to be able to play songs with greater length. In the first approach we generated samples with a sample frequency of 48000Hz. This sample rate produced sounds with relatively high sound quality, but required a substantial amount of storage.

The sound samples were generated with the use of Audacity\cite{audacity} and Switch\cite{switch}. These programs were used to change the size of the samples and adjust the sample rate. The samples were then further extracted from the .wav files, and extracted into an array consisting of 8-bit chars. For this purpose we used a C program by Rafael Vuijk \cite{darkfader}. The sample array could then be then statically linked with the code. However, the relatively low sample rate required some changes in the code. Note that the sound synthesiser used a sample rate of 48000 instead. In order to be able to play the samples at 8000Hz we had to adjust the sample frequency when playing the precomputed samples. This was achieved by simply changing TOP register of the timer. By adjusting this register we were able to change the interrupt frequency, and thereby changing the sample frequency, so that each sample could simply be played at each interrupt.  



  











