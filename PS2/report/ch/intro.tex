\section{Introduction}
The description of the exercise was to use the EFM32GG-DK3750 development board to create different sounds effects. The sound effects are played when buttons on the gamepad controller are pushed, where each button corresponds to a different sound effect. These are realized through the onboard Digital to Analog converter that is connected to an amplifier. The actual sound is created by waves that are created by oscillations. The frequency, period, and amplitude, of the wave determines the sound. 

The main goals of the assignment were to learn the basics of how to program on a microcontroller without the support of an operating system, as well as learn more about the behaviour of sound waves and become more familiar with the C progamming language. An iterative design approach was taken in order to test the program as frequently as possible and make good use of the GNU Debugger (GDB) in order to avoid a complex debugging task at the end of the implementation.



%The sound is created by waves that are created by oscillations. The physical sound is created by the properties of this wave. The frequency, period and amplitude of the wave determines the sound. The waves can be created synthetically, by creating a sound synthesiser in software or by using pre-sampled sound waves.  

%Different sounds can be realised by creating different waves of certain frequency and amplitude. The frequency determines the tone and the amplitude determines the strength of the tone. The length of the tone is determined by the period of that particular wave. 


    

