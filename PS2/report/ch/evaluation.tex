\section{Evaluation}

Due to the fact that neither of the group members had previously programmed in C without the support of an operating system, the learning curve was initially quite high. As a result, this ended up being a challenging assignment, but also a very educational one. When programming with limited resources, one has to think about a lot of constraints that are taken for granted in higher-level programming. Using C for this purpose has also taught both of us more about aspects of the language that were previously unknown, and has also made us realize how powerful the language can be.
	
	We ran into quite a few issues along the way, some which we could never quite figure out. For instance, when flashing the microcontroller, the reset button has to be pressed for the microcontroller to go into low energy mode. This is a similar issue that occured in excercise 1, where a solution has yet to be found. 
	Another major issue was the background noise from the DAC when switching to the low-energy timer. This occured even when disabling the DAC after use. This was something we spent hours trying to solve, until we received consultation from one of the teaching assistants and were told to switch to a different output mode. While this was a frustrating issue, however, the debugging process taught us more about the behaviour of the DAC and its different modes.
	
All in all, this has been a very rewarding assignment. Not only have we gotten more experience in programming on microcontrollers, but we have also learned a lot about the process of sound generation through sound synthesis, sampling, and various waveforms, as this was an area where both of the group members had very little experience. Given more time, one area of improvement would be the sampling and sound quality, as this was an area that had to be cut short when the assignment rapidly approached its due date.