\begin{abstract}

A microcontroller is basically a small computer designed for embedded purposes, for instance for controlling household appliances or regulate sensitive systems such as temperature regulation systems, motion sensors, etc. Hardware specifications, I/O interfaces, and even the size of a microcontroller will vary greatly depending on the manufacturer and its designed purpose, but in the simplest of terms, a regular microcontroller works in such a way that it gets an input, and outputs the processed data on some sort of output port for further use.
	

The unique aspect of the EFM32GG is its focus on energy efficiency, as its potential in low energy consumption outperforms any other available 8-, 16-, and 32-bit solution currently on the market \cite{EFM32GG-rm}. In this report, we analyze these capabilities by writing a program in assembly code using the ARM Thumb2 Instruction Set to control LEDs on an attached gamepad, both hardcoded and using the GPIO buttons. The report will go into detail on how this was done and how different approaches were used to test the total power consumption of the development kit.

The unique aspect of the EFM32GG is its focus on energy efficiency, as its potential in low energy consumption outperforms any other available 8-, 16-, and 32-bit solution currently on the market \cite{EFM32GG-rm}. In this report, we analyze these capabilities by writing a program in assembly code using the ARM Thumb2 Instruction Set to control LEDs on an attached gamepad, both hardcoded and using the GPIO buttons. The report will go into detail on how this was done and how different approaches were used to test the total power consumption of the development kit by observing changes in the development kit energy monitor.


\end{abstract}