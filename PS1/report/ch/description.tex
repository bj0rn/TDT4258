\section{Description And Methodology}

The following section explains the setup of the microcontroller regarding GPIO, buttons and LEDs step by step. \ref{ch:initial_setup}.


\subsection{GPIO clock setup} \label{ch:initial_setup}
The EFM32GG microcontroller is memory mapped, and the I/O devices are accessed by reading and writing to specific memory locations. The CPU uses normal read- and write instructions to communicate with the devices, and each I/O device is given a specific address in the memory map of the device. To save memory, only the specific I/O controllers used are clocked. This implies that the the I/O controller needs to be enabled before it can be used. Fortunately, the EFM32GG has a dedicated I/O controller for handling the different clocks, called the Clock Management Unit (CMU). A specific controller can be enabled by writing to the \emph CMU\_HFPERCLKEN0 register. This register consists of 32 bits where each bit corresponds to an I/O controller.

To interface the gamepad to the EFM32GG, GPIO pins are used. These pins are controlled through the GPIO controller. To enable an I/O controller, the value 1 is written to its corresponding register position. In case of enabling the GPIO clock position, bit 13 is set to 1. The following instructions gives an example how this can be achieved: 

\begin{lstlisting}
ldr r1, =CMU_BASE                //r1 = CMU_BASE
ldr r2, [r1, #CMU_HFPERCLKEN0]   //r2 = CMU_HFPERCLKEN0 

mov r3, #1                       //r3 = 1
lsl r3, r3, #CMU_HFPERCLKEN0_GPIO//Shift left 13 positions
orr r3, r2, r3                   //Enable the GPIO clock

str r2, [r1, #CMU_HFPERCLKEN0]   //Store the register back 
\end{lstlisting}

%Se mer på!!
\subsection{LED setup}\label{led-setup}

The GPIO pins are divided into several different ports on the EFM32GG-DK3750 development board. Each port corresponds to a different location in memory.  The pins are configured by writing to the $GPIO\_MODEL$ and $GPIO\_MODEH$ registers. These can be accessed by using their position relative to the GPIO base of the specific port. These registers contain 8-bit fields that can be used to configure the pins. As described on pages 758-759 in the EFM32GG reference manual, the $GPIO\_Px\_Model$ registers contain 8 bit fields named MODEn (n=0,1, .., 7). These fields control pins 0-7, which in effect controls pins 0-7. The $GPIO\_Px\_MODEH$ registers contain 8 bit fields named MODEn (n=8, 9, ..15), which controls pins 8-15. More information about the specific configurations can be found in said pages of the reference manual.

In the case of the gamepad controller, the pins to the LEDs are connected to port A. These are assigned to output by writing {\emph 0x55555555} to  the $GPIO\_PA\_MODEH$ register. This corresponds to setting pins 8-15 to an "open drain with pull-up" configuration\cite[p.758-759]{EFM32GG-rm}. As a result, the transistors connect to low, and the resistor connects to high. Additionally, the drive strength of the pins are set to high. This is accomplished by writing 0x02 to the $GPIO\_PA\_CTRL$ register. This makes it possible to toggle the LEDs by writing 0 to the upper level of the $GPIO\_PA\_DOUT$ registers, as the LEDs are configured to respond to an active low signal. The following instructions gives an example on how this can be achieved using assembly instructions:


\begin{lstlisting}
ldr r1, =GPIO_PA_BASE        // r1 = GPIO_PA_BASE
ldr r2, =0x55555555          // r2 = 0x55555555  
str r2, [r1, #GPIO_MODEH]    //Store config
mov r2, #0x2                 // r2=0x02
str r2, [r1, #GPIO_PA_CTRL]  // GPIO_PA_CTRL=0x2                    

\end{lstlisting}


 

\subsection{Button setup}\label{ch:buttons}
The buttons are configured in a similar way to the configurations of the LEDs, with some minor differences. The pins are set to input by writing {\emph 0x33333333} to the $GPIO\_PC\_MODEL$ register. This corresponds to configuring each pin to "open drain with alternating drive strength"\cite[p.758-759]{EFM32GG-rm}. Next, the internal pull-up is enabled by writing 0xff to the $GPIO\_PC\_DOUT$ register. The current state of the buttons can be seen by reading the $GPIO\_PC\_DIN$ register. 

An example of the configuration can be viewed in the following code snippet: 

\begin{lstlisting}
ldr r1, =GPIO_PC_BASE        //r1 = GPIO_PC_BASE
ldr r2, =0x33333333          //r2 = 0x33333333  
str r2, [r1, #GPIO_MODEL]    //Store config
mov r2, #0xff                // r2 = 0xff
str r2, [r1, #GPIO_DOUT]     // Internal pull-up

\end{lstlisting}





\subsection{Polling}\label{ch:polling}
The simplest way of communicating with an I/O device is what is known as busy waiting\cite[p. 99]{wolf}. This works in the way that the CPU constantly checks the device to see if some register value(s) have changed. The status of the buttons are checked by reading the button input registers, and if the button is toggled, the program responds by writing a certain value to the LEDs. In the specific case of the gamepad used, this is accomplished by a loop continuously checking the $GPIO\_PC\_DIN$ register. If one of the bits in range 0-7 is set, the button is pushed. Note that the buttons are active low. An example of polling is provided in the following code snippet:

\begin{lstlisting}
     ldr r1, =GPIO_PC_BASE   // r1 = GPIO_PC_BASE
     ldr r2, =GPIO_PA_BASE   // r2 = GPIO_PA_BASE
while:
    ldr r3, [r1, #GPIO_DIN]  // r3 = GPIO_PC_DIN
    lsl r3, r3, #8           // r3 = shift left 8  
    str r3, [r2, #GPIO_DOUT] // GPIO_PA_DOUT = r3
    b while                  // loop
\end{lstlisting}

In this code example, the buttons are continuously polled. The result from each read-cycle is written to the $GPIO\_PA\_DOUT$ register. Since the buttons are located at the lower pins, and the buttons are located at the higher addresses, the value is shifted 8 positions to the left before it is written to the $GPIO\_PA\_DOUT$ register. 

\subsection{GPIO interrupts}\label{ch:interrupts}
The busy waiting approach is very inefficient, as a lot of computational resources are wasted. A far more energy efficient approach is to use interrupts. With this approach, the I/O controllers themselves are responsible for informing the CPU when changes have occurred by sending information regarding type of interrupt. The CPU then stores and handles all relevant states, changes the flow of execution, and sets the program counter to point to the relevant interrupt handler.

The EFM32GG processor, the ARM cortex-M3, contains an interrupt controller that can be used to enable and disable various interrupts. This interrupt controller is known as the Nested Vectored Interrupt Controller(NVIC). This vector table contains the addresses of exception handlers and the Interrupt service routines (ISRs). In the ARM cortex M family of processors, each interrupt requests(IRQ) received has its own ISR. The starting address of each ISR is stored in an interrupt vector table. A lookup in the interrupt handler is performed every time an interrupt request is triggered. The interrupt table contains the service routines for interrupts. \cite{EFM32-int}. Interrupts can be enabled by writing to the \emph ISER0 and \emph ISER1 registers. These registers contain one bit for each interrupt source. An overview of the different interruption sources supported, and their respective bit positions, can be found on pages 12-13 in the EFM32GG reference manual.

There are two interrupt-sources, GPIO\_ODD and GPIO\_EVEN, that handle enabling of interrupts for odd and even GPIO pins, respectively. These are located at positions 1 and 13 in the $ISER0$ register. Since there is no point in distinguishing between even and odd pins in this particular assignment, a better approach is to make sure both interrupt sources corresponds to the same interrupt handler. This can be accomplished by writing 0x802 to the $ISER0$ register, which enables the interrupts for even- and odd-numbered pins.

The registers $GPIO\_EXTIPSELL$ or $GPIO\_EXTIPSELH$ are used to select which port will trigger the interrupt flag. These registers are organized such that all the pins with the same number are grouped together. The $GPIO\_EXTIPSELL$ is responsible for the pins ranging from 0 to 7, and $GPIO\_EXTIPSELH$ are responsible for the pins ranging from 8 to 15. In each of these registers, 3 bits are used to specify which pin should trigger the interrupt flag. In this particular case, the pins are located at port C, and the button pins range from 0 to 7. Therefore, we write the value 0x22222222 to the $GPIO\_EXTIPSELL$ register.

In order to sense the interrupt flag, the registers $GPIO\_EXTRISE$ and/or $GPIO\_EXTFALL$ need to be set. These registers enable sensing of rising and falling edges, respectively. These can be enabled by writing 0xff to both of these registers. To enable interrupt generation, 0xff is written to $GPIO\_IEN$. An example of setting up interrupts for the GPIO-pins is provided in the following code snippet: 

\begin{lstlisting}
ldr r1, =ISER0                //r1 = ISERO
ldr r2, =0x802                //r2 = 0x802  
str r2, r1                    //store pins

ldr r1, =GPIO_BASE            //r1 = GPIO_BASE
ldr r2, =0x222222222          //r2 = 0x222222222
str r2, [r1, #GPIO_EXTIPSELL] //store config

mov r2, 0xff                  //r2 = 0xff 
str r2, [r1, #GPIO_EXTIFALL]  //sense falling edge
str r2, [r1, #GPIO_EXTIRISE]  //sense rising edge
str r2, [r1, #GPIO_IEN]       //Enable interrupt handling 

\end{lstlisting}

When working with interrupts, it is important to note that an interrupt flag is set. If not cleared, this flag will cause the interrupt handler to be continuously called. In order to reset this flag the $GPIO\_IFC$ register needs to be written. This register contains an overview of pending interrupts \cite[p. 763]{EFM32GG-rm}. An interrupt flag is reset by writing its corresponding bit to 0.  The simplest way of achieving this is by by reading the $GPIO\_IF$ register, which contains the source of the interrupt. By writing this value to $GPIO\_IFC$, the flag will be cleared. The following code snip provides an example of how this can be achieved:

\begin{lstlisting}
ldr r1, =GPIO_BASE
ldr r2, [r1, #GPIO_IF]
str r2, [r2, GPIO_IFC]
\end{lstlisting}
 
{\bf need more cleanup}


\subsection{Energy Optimization}\label{ch:energy_optimization} 
The EFM32GG microcontroller has a rich set of features regarding improving energy efficiency, the main focus being on turing off components that aren't used or simply decreasing power consumption by tuning their performance. When performing these optimization, one must be careful considering different trade-offs \cite{EFM32GG-energy}. However, to fully utilise the energy optimizations, one must also consider the code. A lot of energy can actually be saved by careful structuring of code. How the code is written and which methods are used directly affect the CPU and memory system of the microcontroller. For instance, it is far better to replace wait loops with interrupts. It is also possible to optimize the memory access. If a program is able to finish earlier, the clock cycles saved can be used in further optimizations. Note that accessing external memory requires more cycles than accessing registers and caches. The EFM32GG is actually delivered with an instruction cache to increase performance while fetching instructions. It is possible to turn this cached off. However, when using a small number of instructions, a far better approach is to leave this on. With small programs like the one presented in this exercise, a chance of a cache hit is large.

The gains from performing these optimizations are limited without combining them with other energy optimization techniques. For convenience, the EFM32GG microcontroller contains a set of different energy levels, where each energy level has certain functionality of the microcontroller available. This allows the programmer to effectively use different energy levels at any given time, depending on the resources actually needed. The transition between the different levels is also optimized to be responsive.  These energy modes are numbered from EM0 to EM4 \cite{EFM32GG-energy}. This makes them ideal when performing operations that does not utilize some of the functionality. For instance, if a component is not doing anything useful, it is better to turn it off while idling. The advantage with the energy levels is that EFM0 to EFM3 are able to sense interrupts.   

The EFM32GG microcontroller allows a very easy transition between the levels using different control registers. In the purpose of this exercise, the focus will be directed at energy levels EFM2 and EFM3. In these modes, the CPU enters a state known as deep sleep mode. Most of the functionality, such as high frequency oscillators and the flash memory, is turned off during this time. During deep sleep mode it is possible to achieve power consumption as low as 900 nA. The CPU will wake up when it receives an interrupt, and go back to sleep when it is finished handling the interrupt.

Entering deep sleep mode can be achieved by writing to the System Control Register(SCR). This register contains two bits of particular interest: \emph DEEPSLEEP and \emph SLEEPONEXIT. When enabled, the DEEPSLEEP bit allows the CPU to enter deep sleep mode. SLEEPONEXIT, when enabled, allows the CPU to enter deep sleep mode when finished handling an interrupt. Both bits can be set by writing the value 6 to the \emph SCR register. When the \emph SCR register is configured, deep sleep mode can be achieved by calling the \emph WFI instructions. This instruction causes immediate entry to sleep mode. 

\begin{lstlisting}
ldr r1, =SRC     //r1 = SRC
mov r2, #6       //r2 = 6         
str r2, r1       // stor config  
\end{lstlisting}

     

The EFM32GG microcontroller also allows are more fine grained approach of tuning the energy consumption. By the far most interesting of these is the possibility of tuning the clocks. The microcontroller has a dedicated clock manager to be able to scale the different clocks to fit the need of the application. Different frequencies imply different energy consumptions. The micro controller also allow the usage of different clocks for different purposes. For a small assignment like this, it is however, difficult to get any beneficial results by doing so. For larger projects, however, this kind of optimizations could be beneficial. Larger projects would typically consist of several different parts with different energy needs. 

{\bf mention  tradeoffs}
{\bf mention ram and other components}   



%% However, it is also possible to modify the components themselves to use less energy. For instance, the clock manager is very powerful and allows the code to scale the clock to different frequencies. It is also possible to setup different clocks for different purposes. It is also possible to turn off parts of the RAM in order to improve energy efficiency. \cite{EFM32-energy} 

  
\subsection{User control}

The second part of the assignment consisted of building on the setup so far to let the user control the LEDs in some way by pressing the gamepad buttons.

\subsubsection{Delay function}

At a clock frequency of 48 MHz, the CPU can potentially process tens of millions of instructions per second. As a result, without some sort of a delay function, any fancy LED display would be processed faster than a human eye could register. For this reason, a delay-function had to be implemented and executed between each change in LED behaviour.

This function was implemented by loading a specified value to a register. The function would then enter a loop, which worked in the way that the value 1 would be subtracted from the value in specified register, and then check the negative status flag. If the value of the flag was set to 1, the program would branch back to the LED display function in question. If the value was set to 0, the function would branch back to the beginning of the loop and repeat. The following code snippet illustrates how this was implemented:

\begin{lstlisting}

delay:
        mov r3, r6 // copy the delay-value to r3 in order to avoid overwriting it
do_wait:
        subs r3, #1 // r3 contains the specified delay value
        bne do_wait
        pop {r6}
        BX lr
        
\end{lstlisting}

The functionality of the delay function was further extended by letting a user set a delay-variable to a certain amount of milliseconds. Through some calculations as well as trial and error, it was figured out that, given the current implementation above, one millisecond would correspond to approximately 5000 $do\_wait$ cycles. In order to convert the delay-variable to the correct amount of cycles, all that needed to be done was multiply said variable by 5000. This was implemented as shown here:

\begin{lstlisting}
convert_to_ms:
        ldr r5, =DELAY //DELAY-variable specified in milliseconds
        ldr r3, =5000
        mul r6, r5, r3
\end{lstlisting}


\subsubsection{Blink}

The purpose of this function was to switch between the four rightmost LEDs being turned on, and the four leftmost LEDs being turned on. With an already-functioning delay function, this was rather easy to implement. One register contained the hexadecimal value for setting the four rightmost LEDs, and another register for the four leftmost LEDs. The value of the first register would then be stored in the $GPIO\_PA\_BASE$ address with $GPIO\_DOUT$ as its offset. The function would then branch to the delay function and run for the time specified by the user, and then repeat the process with the second register. The function would then branch back to the beginning and repeat. The following code illustrates how this was implemented:
\begin{lstlisting}

mov r2, #0x0f
mov r4, #0xf0
lsl r2, r2, #8
lsl r4, r4, #8
while_loop:
        str r2, [r1, #GPIO_DOUT] // r1 contains the GPIO_PA_BASE address
        BL delay
        str r4, [r1,#GPIO_DOUT]
        BL delay
        B while_loop


\end{lstlisting}

Although not very significant regarding the total energy consumption, the move and logical left shift instructions were placed before the while loop in order to avoid unnecessary instruction redundancies.

\subsubsection{Wave}
The wave function is implemented to give feedback from the buttons. The function works by creating a wave effect from the button pushed. When a button is pushed the other LEDs to left is turned on one by one, with some specified delay. 

The wave effect is accomplished by using XOR to isolate the position of the button. By shifting this value to the right the value can be XORed with the original button state, thus turning on the LED left to it. By doing this in a loop with the delay function, a wave like effect is achieved. The loop is finished when the value of the bit mask reaches 0. The following code snippet illustrates how this can be achieved. 

\begin{lstlisting}
ldr r1, =GPIO_PA_BASE         //r1 = GPIO_PA_BASE
ldr r2, =GPIO_PC_BASE         //r2 = GPIO_PC_BASE
ldr r3, [r1, #GPIO_DOUT]      //Load button state
mov r4, #0xff               
eor r5, r3, r4                //Create mask 
mov r9, #0

light:
    eor r3, r5, r3           //Enable next bit  
    bl delay                 //Delay
    lsl r11, r3, #8          //Shift 8   
    str r11, [r1, #GPIO_DOUT]//Turn on LEDs 
    bl delay                 //Delay
    lsr r5, r5, #1           //Shift 1 right 
    cmp r9, r5               //Reached end of loop ? 
    bne light
\end{lstlisting}

 












