\subsection{Background}
A device driver is a small program that interfaces some external hardware. The driver provides a software device to the hardware, enabling operating systems and other computer programs to access hardware functions without needing details of the hardware. It can be viewed as an abstraction to make interaction between software and hardware easier. The program acts as a translator between the hardware and the software layer. 

Device drivers usually require low-level access to hardware functions in order to operate. They typically operate in a highly privileged environment. Note that a normal computer operating system is divided into different security layers. Each layer corresponds to a set of rules specifying the mode of operation. These layers are enforced by the operating system and supported by the CPU. This report will not focus much on these levels, other than introducing {\bf Kernel Mode} and {\bf User Mode}\cite{linux}. 

\subsubsection{User mode vs kernel mode}
An operating system has two different modes: kernel mode and user mode. In user mode, code cannot directly access hardware or reference memory, and must do so by delegating to system APIs. Running code in user mode is considered quite safe, and crashes are almost always recoverable. Due to this, most of the code on a computer runs in user space. 

In kernel mode, the code has access to all underlying hardware and can access all parts of memory. In other words, there are no restrictions in kernel space. This means that kernel mode crashes can potentially be catastrophic, and only the most trusted functions of an operating system runs in kernel mode. 

As mentioned, one can only access hardware or reference memory through system APIs. In many cases, these APIs are device drivers, and work as an interface between kernel and user mode. These drivers are the main topic of this assignment.



