\section{Discussion}

\subsection{Tickless idle}
One approach of energy optimization is configuring the operating system to consume less power. It is possible to tweak the kernel configuration to increase the idle time and/or use the time idle more efficiently. One way of achieving this is introducing tickless idle. Linux is configured with a timer that is periodically invoked in order to handle OS tasks like process accounting, scheduler load balancing, and maintaining per-CPU timer events. The disadvantage with this approach is that these timer ticks occur regardless off CPU state. This implies that even if the CPU is in idle state it will be interrupted by these ticks. By enabling "tickless idle", linux can eliminate this periodic behaviour when the CPU is idle. This allow the CPU to be in a power saving state a longer period of time, thereby decreasing the overall power consumption\cite{tick}. This approach was attempted in our implementation. However, the gains were quite limited. The main problem in the code is that the time spent in idle mode is not sufficient to make a significant difference. For this optimization to be beneficial, the program needs to spend more time in idle mode. This is quite difficult to achieve considering that the game is in constant movement. 

\subsection{The linux driver}
The driver explained in the previous sections employ the old techniques for developing a driver. In a modern approach, the driver should be structured in another way. Modern linux drivers are implemented as platform drivers. In the old approach, the driver performs most of its initialization in the init function, and assumes that the hardware is located at some specific address. The new way of structuring drivers does not make such assumptions, but is made more dynamic instead. The driver simply informs the kernel what kind of hardware it can handle. Then it waits until the kernel decides to activate the driver. The kernel only activates the device driver if the device exists. When the driver decides to activate the driver, all information about the device is gathered through querying the device. The advantage with this method is that the driver becomes more generic. It does not rely that much on hardcoded addresses. The platform method is realised by moving the initializing process of the driver to a set of probing functions. 

We have not implemented this approach. We found it more important to understand the old approach of developing a linux driver. This due that fact that the curriculum covers this approach in great depth. The same principles apply in the platform approach, and it should be easy to re-write the driver to a platform based driver. However, the conversion of the driver was considered low priority.


\subsection{The Game}
The Pong is game where the ball is in constant movement. This makes the energy optimizations particular hard. For each movement of the ball we need to perform a large range of collision detection tests. This implies that the CPU is utilised most of the time, which affects the energy consumption of the program. We have tried to limit the time of computation by optimizing the operations in the code. A greater focus on energy efficiency could have been achieved by developing a game that is updated less frequently. 

However, it should be possible to decrease the power consumption on the existing game by optimizing the operations in the game even further, such as removing the use of floating point numbers. This would able us to reduce the time spent performing computations, and allow the CPU to sleep more frequently. 