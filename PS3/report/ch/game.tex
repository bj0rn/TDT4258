\subsection{Pong}
A pong game was developed in order to communicate with the gamepad driver. The game had to be kept relatively simple, as the resources on the EFM32GG-DK3750 are quite limited.

\subsubsection{Game description}

The concept of the game is simple: There are two paddles, each controlled by a player, and a ball moving back and forth. The leftmost paddle is controlled by player 1, and the rightmost paddle by player 2. If the ball collides with the wall, i.e. a player is not able to reach the ball with the paddle, the opposing player is awarded a point. The first player to reach 10 points wins. The code for the game is divided into two main files: display.c, which implements all the functions updating various parts of the display screen, and game.c, which contains the game logic as well as the main function loop for running the game.

As the main focus of this exercise is the device driver, not too much focus is put on giving an in-detail description of the game implementation. Nonetheless, below is a brief description of the main parts of the game.

\subsubsection{Game logic}

As mentioned, most of the game logic lies in the game.c-file. The functions in this file are responsible for opening the gamepad driver, handling gamepad input and updating the ball and paddle positions. This is done in functions such as \emph{$init\_gamepad$}, \emph{$map\_buttons$}, \emph{$move\_ball$} and \emph{$move\_paddle$}. \emph{$*intersect\_rectangle\_circle$} is responsible for detection collisions between the ball and the paddles, and thus adjusting the direction and speed of the ball accordingly.

\subsubsection{Display logic}

The logic for updating the various parts of the display lies in the display.c-file. The idea is fairly simple: open the framebuffer driver (fb0) on initialization, and memory map the amount of screen bytes to an array with the use of the \emph{mmap}-function. Each pixel on the screen would then be represented by two bytes by using the RGB color model (with five, five, and six bits respectively). Writing a 16-bit RGB value to a position in the memory mapped array would then set the pixel to the appropriate color, and the screen would then be updated when refreshing the screen with the \emph{ioctl}-command. The paddles and ball are represented as structs with attributes such as length, height, and color.

\subsubsection{Images}

Other than the game itself, an intro screen and an ending screen as well as display for showing the game score were created. The original idea was to create 320x240px images in a .tga format and load them onto the screen. However, we had a lot of issues with this, as not only would the images not load correctly, but also, the amount of the spent refreshing the screen after trying to load the images took an unreasonable amount of time. Therefore, we decided to create our own images instead by creating arrays of chars, where each char would represent a 10x10px area on the screen and a specific color ('X'=black, 'W'=white, etc...). Functions such as \emph{$draw\_rect$}, \emph{$draw\_score$}, and \emph{$draw\_text$} were responsible for display these on the appropriate positions on the screen. The char-arrays themselves can be viewed in the \emph{fonts.h}-file.


\subsubsection{Optimizations}

Due to the limited resources on the EFM32GG-DK3750, certain optimization techniques had to be used. For instance, when drawing circular objects, such as the pong ball, one would normally use a square root function, which demands a lot of computational resources. Instead of using a traditional circle-drawing algorithm, an 8-way symmetry algorithm was implemented instead \cite{circle-algorithm}. This algorithm requires only 1/4 as much computational power as a naïve circle drawing algorithm, which in length saves a lot of resources as the circle is drawn many times per second.
	Another optimization technique used was to only redraw the part of the screen that had changed from one instance to another, so that the entire screen wouldn't be redrawn every time something on the map were to change.  

